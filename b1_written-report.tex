% Options for packages loaded elsewhere
\PassOptionsToPackage{unicode}{hyperref}
\PassOptionsToPackage{hyphens}{url}
\PassOptionsToPackage{dvipsnames,svgnames,x11names}{xcolor}
%
\documentclass[
  letterpaper,
  DIV=11,
  numbers=noendperiod]{scrartcl}

\usepackage{amsmath,amssymb}
\usepackage{iftex}
\ifPDFTeX
  \usepackage[T1]{fontenc}
  \usepackage[utf8]{inputenc}
  \usepackage{textcomp} % provide euro and other symbols
\else % if luatex or xetex
  \usepackage{unicode-math}
  \defaultfontfeatures{Scale=MatchLowercase}
  \defaultfontfeatures[\rmfamily]{Ligatures=TeX,Scale=1}
\fi
\usepackage{lmodern}
\ifPDFTeX\else  
    % xetex/luatex font selection
\fi
% Use upquote if available, for straight quotes in verbatim environments
\IfFileExists{upquote.sty}{\usepackage{upquote}}{}
\IfFileExists{microtype.sty}{% use microtype if available
  \usepackage[]{microtype}
  \UseMicrotypeSet[protrusion]{basicmath} % disable protrusion for tt fonts
}{}
\makeatletter
\@ifundefined{KOMAClassName}{% if non-KOMA class
  \IfFileExists{parskip.sty}{%
    \usepackage{parskip}
  }{% else
    \setlength{\parindent}{0pt}
    \setlength{\parskip}{6pt plus 2pt minus 1pt}}
}{% if KOMA class
  \KOMAoptions{parskip=half}}
\makeatother
\usepackage{xcolor}
\setlength{\emergencystretch}{3em} % prevent overfull lines
\setcounter{secnumdepth}{-\maxdimen} % remove section numbering
% Make \paragraph and \subparagraph free-standing
\makeatletter
\ifx\paragraph\undefined\else
  \let\oldparagraph\paragraph
  \renewcommand{\paragraph}{
    \@ifstar
      \xxxParagraphStar
      \xxxParagraphNoStar
  }
  \newcommand{\xxxParagraphStar}[1]{\oldparagraph*{#1}\mbox{}}
  \newcommand{\xxxParagraphNoStar}[1]{\oldparagraph{#1}\mbox{}}
\fi
\ifx\subparagraph\undefined\else
  \let\oldsubparagraph\subparagraph
  \renewcommand{\subparagraph}{
    \@ifstar
      \xxxSubParagraphStar
      \xxxSubParagraphNoStar
  }
  \newcommand{\xxxSubParagraphStar}[1]{\oldsubparagraph*{#1}\mbox{}}
  \newcommand{\xxxSubParagraphNoStar}[1]{\oldsubparagraph{#1}\mbox{}}
\fi
\makeatother


\providecommand{\tightlist}{%
  \setlength{\itemsep}{0pt}\setlength{\parskip}{0pt}}\usepackage{longtable,booktabs,array}
\usepackage{calc} % for calculating minipage widths
% Correct order of tables after \paragraph or \subparagraph
\usepackage{etoolbox}
\makeatletter
\patchcmd\longtable{\par}{\if@noskipsec\mbox{}\fi\par}{}{}
\makeatother
% Allow footnotes in longtable head/foot
\IfFileExists{footnotehyper.sty}{\usepackage{footnotehyper}}{\usepackage{footnote}}
\makesavenoteenv{longtable}
\usepackage{graphicx}
\makeatletter
\def\maxwidth{\ifdim\Gin@nat@width>\linewidth\linewidth\else\Gin@nat@width\fi}
\def\maxheight{\ifdim\Gin@nat@height>\textheight\textheight\else\Gin@nat@height\fi}
\makeatother
% Scale images if necessary, so that they will not overflow the page
% margins by default, and it is still possible to overwrite the defaults
% using explicit options in \includegraphics[width, height, ...]{}
\setkeys{Gin}{width=\maxwidth,height=\maxheight,keepaspectratio}
% Set default figure placement to htbp
\makeatletter
\def\fps@figure{htbp}
\makeatother

\KOMAoption{captions}{tableheading}
\makeatletter
\@ifpackageloaded{caption}{}{\usepackage{caption}}
\AtBeginDocument{%
\ifdefined\contentsname
  \renewcommand*\contentsname{Table of contents}
\else
  \newcommand\contentsname{Table of contents}
\fi
\ifdefined\listfigurename
  \renewcommand*\listfigurename{List of Figures}
\else
  \newcommand\listfigurename{List of Figures}
\fi
\ifdefined\listtablename
  \renewcommand*\listtablename{List of Tables}
\else
  \newcommand\listtablename{List of Tables}
\fi
\ifdefined\figurename
  \renewcommand*\figurename{Figure}
\else
  \newcommand\figurename{Figure}
\fi
\ifdefined\tablename
  \renewcommand*\tablename{Table}
\else
  \newcommand\tablename{Table}
\fi
}
\@ifpackageloaded{float}{}{\usepackage{float}}
\floatstyle{ruled}
\@ifundefined{c@chapter}{\newfloat{codelisting}{h}{lop}}{\newfloat{codelisting}{h}{lop}[chapter]}
\floatname{codelisting}{Listing}
\newcommand*\listoflistings{\listof{codelisting}{List of Listings}}
\makeatother
\makeatletter
\makeatother
\makeatletter
\@ifpackageloaded{caption}{}{\usepackage{caption}}
\@ifpackageloaded{subcaption}{}{\usepackage{subcaption}}
\makeatother

\ifLuaTeX
  \usepackage{selnolig}  % disable illegal ligatures
\fi
\usepackage{bookmark}

\IfFileExists{xurl.sty}{\usepackage{xurl}}{} % add URL line breaks if available
\urlstyle{same} % disable monospaced font for URLs
\hypersetup{
  pdftitle={Written Report},
  colorlinks=true,
  linkcolor={blue},
  filecolor={Maroon},
  citecolor={Blue},
  urlcolor={Blue},
  pdfcreator={LaTeX via pandoc}}


\title{Written Report}
\author{}
\date{}

\begin{document}
\maketitle


\subsection{Introduction and Data}\label{introduction-and-data}

Urbanization shapes the demographic and economic conditions of states
across the U.S., influencing population structure and workforce
characteristics. Our project examines how levels of urbanization may
relate to the average age of working-age residents. To explore this
relationship, we compare California, the most urbanized state at 94.2\%,
with Vermont, the least urbanized state at 35.1\%. Understanding whether
less urbanized areas have older working populations can offer insight
into potential economic challenges, such as shrinking labor forces or
reduced productivity.

Our analysis uses data from the Current Population Survey (CPS), a
monthly survey conducted by the U.S. Census Bureau. The CPS includes
demographic and labor-force information such as age, state, and
employment status, but for this project we focus only on respondents'
age and state of residence. We restricted the dataset to individuals
living in California or Vermont to directly compare their working-age
populations. This leads us to our research question: \textbf{Is the
average age of Vermont's working-age population higher than that of
California?}

\subsection{Methodology}\label{methodology}

The data for this project was obtained from the CPS library in R, stored
in our project's ``data'' folder as data.qmd. We filtered this dataset
to include only residents of California and Vermont, resulting in 377
observations across 17 variables. For our analysis, we focused
exclusively on the age and state variables to examine the average age of
the working-age population in each state.

To test our research question, we conducted a one-sided t-test. In a
one-sided test, the null hypothesis assumes that the parameter of
interest is greater than or equal to a specific value, while the
alternative hypothesis assumes it is less. This approach was appropriate
because our study aims to determine whether the average age in Vermont
is higher than in California, rather than simply testing for any
difference. The t-test was conducted by creating a combined variable for
California and Vermont (vt\_ca) and testing the difference in means with
the null hypothesis that the mean age difference is less than or equal
to zero. Additionally, we visualized the age distributions of the two
states using box and whisker plots. Separate plots were created for
California and Vermont, with age represented on the y-axis. These plots
allow for a visual comparison of the central tendency and spread of ages
within each state, highlighting differences in distributions and
variability.

\subsection{Results}\label{results}

The one-sided t-test comparing the average age of the working-age
population in Vermont and California showed that Vermont has a higher
mean age than California. The test provided statistical evidence to
reject the null hypothesis, supporting the conclusion that Vermont's
workforce is older on average.

The box and whisker plots further illustrate the differences between the
two states. Vermont's age distribution is skewed toward older ages, with
a wider interquartile range, indicating a larger proportion of residents
above the mean age. In contrast, California's distribution is more
compressed and centered around a slightly lower mean, reflecting a
younger and more evenly distributed workforce.

Together, the t-test and visual analysis confirm that urbanization
appears associated with differences in workforce age, with the less
urbanized Vermont showing an older working population compared to the
highly urbanized California.

\subsection{Discussion \& Conclusion}\label{discussion-conclusion}

Our analysis indicates that Vermont, the less urbanized state, has a
higher average age among its working-age population compared to
California, the most urbanized state. This finding aligns with the
expectation that lower urbanization may be associated with older
populations, potentially due to lower residents migrating toward urban
areas for education and employment opportunities. The difference in age
distributions, as shown in the box and whisker plots, highlights how
demographic trends vary with urbanization levels and may have
implications for workforce availability and economic growth.

These results suggest that states with lower urbanization may face
challenges related to an aging workforce, such as labor shortages,
reduced productivity, or increased pressure on social services.
Conversely, highly urbanized states like California may benefit from a
younger and more evenly distributed workforce, which could support
sustained economic activity and innovation.

In conclusion, our project demonstrates a clear relationship between
urbanization and workforce age, showing that less urbanized areas tend
to have older working populations. Understanding these demographic
patterns is important for policymakers and planners, as it can inform
strategies to attract younger workers, support aging populations, and
maintain economic stability across states with varying levels of
urbanization.

\subsection{Outline \& Breakdown of Our
Project}\label{outline-breakdown-of-our-project}

Our primary project work and data analysis can be found in
\textbf{Project.Rproj.qmd}

This file contains our \textbf{hypothesis testing}, and was where we
were able to determine whether to reject or accept the null
hypothesis\textbf{.} Here, you can finding the following information:

\begin{enumerate}
\def\labelenumi{\arabic{enumi}.}
\tightlist
\item
  CPS Library Pull 2. One-Sided T-Test 3. Box and Whisker Plots 4.
  Confidence Intervals 5. T-Distribution Plot w/ Critical Value
\end{enumerate}

\textbf{Break down of each of the following is below:}

\textbf{CPS Library Pull:}

\textbf{What is this?}

The data for our project is stored within an R Library. This data set
can be found isolated in our ``data'' folder in data.qmd. To collect the
data we needed for this project, we filtered the data set by two states:
\textbf{CA and VT}. This data set provided us with 377 rows and 17
columns of information.

\textbf{One-Sided Test:}

\phantomsection\label{T-Test}
\textbf{What is a one-sided Test?}

\emph{For a one-sided test of an unknown parameter, the null and
alternative hypothesis are H0: Theta \textgreater= c, \& H1: Theta
\textless{} c, respectively. The hypothesis test of H0 determines
whether there is statistical evidence to reject H0.}

\textbf{Why was this appropriate for our research?}

\emph{Provided that we are seeking to compare two different values---the
average age of a workforce eligible population--this is more appropriate
than a two-sided test. For a two-sided test, the null hypothesis is
instead H0: theta = c, and the alternative hypothesis is H1 != c.~This
form of testing is not appropriate given our research topic.}

\textbf{How did we conduct the T-Test?}

\emph{To conduct the t-test we isolated the Vermont and California CPS
data and creates a variable ``vt\_ca''. The t-test was ran with the null
hypothesis that the difference in means between California and Vermont
was less than 0, which would indicate that the mean age of Vermont is
greater than that of California. The alternative hypothesis was
``greater''.}

\textbf{Box and Whisker Plots:}

\textbf{What box and whisker plots did we have?}

\emph{We used two box and whisker plots, one representing the age
distribution in California and the other representing the age
distribution in Vermont.}

\textbf{How did we create the box and whisker plots?}

\emph{We created the box and whisker plots in R by isolating the Vermont
and California age data. Each plot only has the data from one state. The
y-axis shows the age of residents while the x-axis is unlabeled as each
plot only measures one variable.}

\textbf{What did the box \& whisker plots tells us?}

\emph{The box and whisker plots show the overall age distribution of
each state. The California plot shows that the mean age is slightly
lower than that of Vermont, and the IQR is more compressed indicating a
more even distribution. The Vermont plot is more skewed towards older
ages, indicating there are more residents around and above the mean age
given.}

\includegraphics[width=4.8125in,height=\textheight]{b1_written-report_files/mediabag/000010.png}

\includegraphics[width=4.8125in,height=\textheight]{b1_written-report_files/mediabag/0000101.png}

\textbf{T-Distribution Plot w/ Critical Value:}

\textbf{What does a T-distribution Plot and CV tell us?}

\emph{A t-distribution plot visually shows the range of possible
t-values under the null hypothesis and how likely each value is. The
critical value (CV) marks the threshold beyond which we would reject the
null hypothesis at a chosen significance level. If the calculated
t-statistic falls past the CV, it indicates the observed difference is
unlikely due to chance. Together, the plot and CV help visually and
quantitatively determine whether to reject the null hypothesis.}

\textbf{How did we show this as a plot?}

\emph{We displayed our results using a t-distribution plot in R by
plotting the probability density function of the t-distribution with
40.22 degrees of freedom. The critical value for our one-sided test at
the 0.05 significance level was marked with a red dashed line. The
calculated t-statistic (-0.448) was indicated with a solid blue line on
the plot. This visual clearly shows that the t-statistic falls far below
the critical value, illustrating that we do not have enough evidence to
reject the null hypothesis.}

\textbf{What did this plot tell us?}

\emph{The t-distribution plot shows the t-statistic of -0.448, which
reflects that the mean age in California (44.51) is slightly less than
that in Vermont (45.31). Because the t-value is negative, it aligns with
the null hypothesis that the difference in means (CA-VT) is less than or
equal to zero. The high p-value (0.6719) indicates that this difference
is not statistically significant, meaning the null hypothesis is more
likely to be true. Overall, the plot visually confirms that we do not
have enough evidence to support the alternative hypothesis that
California's working-age population is older than Vermont's.}

\includegraphics{b1_written-report_files/mediabag/000014.png}




\end{document}
